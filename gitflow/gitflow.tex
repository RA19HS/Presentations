\documentclass[10pt]{beamer}
\mode<beamer>{%
	\usetheme[hideothersubsections,right,width=22mm]{Rochester}
	\usecolortheme{crane}
}
\title{Gitflow}
\author[R. A.]{Reza A.\\[2mm]\tiny{\texttt{ra19@tutanota.com}}}
\date{Jul 23 2022}
\begin{document}
	\begin{frame}
		\titlepage
	\end{frame}
	\section{Introduction}
	\begin{frame}
		\frametitle{Git Workflows}
		\begin{definition}
			A \alert{Git workflow} is a recipe or recommendation for how to use Git.
		\end{definition}
		\pause
		\begin{example}
			\begin{itemize}
				\item[$\lozenge$] Centralized workflow \pause
				\item[$\lozenge$] Feature branch workflow \pause
				\item[$\lozenge$] \alert{Gitflow} workflow \pause
				\item[$\lozenge$] Forking workflow
			\end{itemize}
		\end{example}
	\end{frame}
	\section{Branches}
	\subsection{Overview}
	\begin{frame}
		\frametitle{Gitflow Branches}
		\begin{block}{Branches}
			\begin{itemize}
				\item[$\lozenge$] Main
				\item[$\lozenge$] Develop
				\item[$\lozenge$] Features
				\item[$\lozenge$] Release
				\item[$\lozenge$] Hotfix
			\end{itemize}
		\end{block}
	\end{frame}
	\subsection{Main and Develop}
	\begin{frame}
		\frametitle{Gitflow Branches}
		\framesubtitle{Main and Develop Branches}
		\includegraphics<1>[width=7.7cm]{main.png}
	\end{frame}
	\subsection{Features}
	\begin{frame}
		\frametitle{Gitflow Branches}
		\framesubtitle{Features Branches}
		\includegraphics<1>[width=7.7cm]{features.png}
	\end{frame}
	\subsection{Release}
	\begin{frame}
		\frametitle{Gitflow Branches}
		\framesubtitle{Release Branch}
		\includegraphics<1>[width=7.7cm]{release.png}
	\end{frame}
	\subsection{Hotfix}
	\begin{frame}
		\frametitle{Gitflow Branches}
		\framesubtitle{Hotfix Branch}
		\includegraphics<1>[width=7.7cm]{hotfix.png}
	\end{frame}
	\section{Example}
	\subsection{Feature}
	\begin{frame}[fragile]
		\frametitle{Gitflow in Command-Line}
		\framesubtitle{Develop and Feature Branches}
		\begin{columns}[t]
			\column{.5\textwidth}
			\begin{block}{Creation}
				\small
				\begin{verbatim}
$ git checkout main
$ git checkout -b develop
$ git checkout -b feature
				\end{verbatim}
			\end{block}
			\column{.5\textwidth}
			\begin{block}{Conclusion}
				\small
				\begin{verbatim}
$ git checkout develop
$ git merge feature
$ git checkout main
$ git merge develop
$ git branch -d feature
				\end{verbatim}
			\end{block}
		\end{columns}
	\end{frame}
	\subsection{Hotfix}
	\begin{frame}[fragile]
		\frametitle{Gitflow in Command-Line}
		\framesubtitle{Hotfix Branch}
		\begin{columns}[t]
			\column{.5\textwidth}
			\begin{block}{Creation}
				\small
				\begin{verbatim}
$ git checkout main
$ git checkout -b hotfix
				\end{verbatim}
			\end{block}
			\column{.5\textwidth}
			\begin{block}{Conclusion}
				\small
				\begin{verbatim}
$ git checkout develop
$ git merge hotfix
$ git checkout main
$ git merge hotfix
$ git branch -d hotfix
				\end{verbatim}
			\end{block}
		\end{columns}
	\end{frame}
	\section{Reference}
	\begin{frame}
		\frametitle{Reference}
		atlassian.com/git/
	\end{frame}
	\begin{frame}
	\end{frame}
\end{document}
